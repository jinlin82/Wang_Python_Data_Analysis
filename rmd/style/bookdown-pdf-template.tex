%\documentclass[UTF8,a4paper,12pt]{znufecls}  % Latex 去掉上面的语句,加上本语句
\documentclass[doctor,twoside,chapterhead,otf]{znufethesis}
%\usepackage{gbt7714}

%======================== 根据选项设置代码处理方式 RMARKDOWN 中独有=============
$if(listings)$
\usepackage{listings}
$endif$

$if(lhs)$
\lstnewenvironment{code}{\lstset{language=R,basicstyle=\small\ttfamily}}{}
$endif$

$if(highlighting-macros)$
$highlighting-macros$
$endif$

$if(verbatim-in-note)$
\usepackage{fancyvrb}
\VerbatimFootnotes % allows verbatim text in footnotes
$endif$

\providecommand{\tightlist}{\setlength{\itemsep}{0pt}\setlength{\parskip}{0pt}}
\newcommand{\passthrough}[1]{\lstset{mathescape=false}#1\lstset{mathescape=true}}

%%% Change title format to be more compact
\usepackage{titling}

% Create subtitle command for use in maketitle
\newcommand{\subtitle}[1]{
  \posttitle{
    \begin{center}\large#1\end{center}
    }
}

\setlength{\droptitle}{-2em}

$if(title)$
  \title{\LARGE\textbf{$title$}}
  \pretitle{\vspace{\droptitle}\centering\huge}
  \posttitle{\par}
$else$
  \title{}
  \pretitle{\vspace{\droptitle}}
  \posttitle{}
$endif$

$if(subtitle)$
\subtitle{$subtitle$}
$endif$

$if(author)$
  \author{$for(author)$$author$$sep$ \\ $endfor$}
  \preauthor{\centering\large\emph}
  \postauthor{\par}
$else$
  \author{}
  \preauthor{}\postauthor{}
$endif$

$if(date)$
  \predate{\centering\large\emph}
  \postdate{\par}
  \date{$date$}
$else$
  \date{}
  \predate{}\postdate{}
$endif$


%% ===============================================================
\begin{document}

$if(title)$
\maketitle
$endif$
\blankpage


$for(include-before)$
$include-before$
$endfor$

%% ========================== Frontmatters ========================
%% -------------- 原创声明-------------
\cleardoublepage
\phantomsection %
\pdfbookmark{声明}{creative}
\input{./misc/creative}
\blankpage

%% ------------- 符号列表-------------
\cleardoublepage
\frontmatter
\pagestyle{plain}

\cleardoublepage
\phantomsection %
\pdfbookmark{\denotationname}{denotation}
\pagenumbering{arabic}
\input{./misc/denotation} % 如果要生成符号列表
\ifodd\thepage
\blankpage
\fi

%% ---------------- 摘要 -------------
\cleardoublepage
\frontmatter
\input{./misc/abstract}
\ifodd\therealpage
\blankpage
\fi

$if(abstract)$
\begin{abstract}
$abstract$
\end{abstract}
$endif$

%% ---------------- 目录 ------------
$if(toc)$
{
$if(colorlinks)$
\hypersetup{linkcolor=$if(toccolor)$$toccolor$$else$black$endif$}
$endif$

\setcounter{tocdepth}{$toc-depth$}
\cleardoublepage
\phantomsection %
\pdfbookmark{\contentsname}{toc}
\pagenumbering{arabic}
\tableofcontents        % 生成目录

\ifodd\thepage
\blankpage
\fi
}
$endif$

$if(lot)$
\cleardoublepage
\phantomsection %
\pdfbookmark{表目录}{tabletoc}
\listoftables           % 如果要生成表目录

\ifodd\thepage
\blankpage
\fi
$endif$

$if(lof)$
\cleardoublepage
\phantomsection %
\pdfbookmark{图目录}{figtoc}
\listoffigures          % 如果要生成图目录

\ifodd\thepage
\blankpage
\fi
$endif$
%% ==========================================================

\mainmatter
\pagestyle{mpage}

$body$

\ifodd\thepage
\blankpage
\fi

%% ========================= 参考文献 =======================
\cleardoublepage
\pagestyle{emptypage}
\renewcommand{\chapterlabel}{\bibname}
\bibliographystyle{./style/gbt7714-plain-zuel.bst}

$if(natbib)$
$if(bibliography)$
$if(biblio-title)$
$if(book-class)$
\renewcommand\bibname{$biblio-title$}
$else$
\renewcommand\refname{$biblio-title$}
$endif$
$endif$
\bibliography{$for(bibliography)$$bibliography$$sep$,$endfor$}

$endif$
$endif$

$if(biblatex)$
\printbibliography$if(biblio-title)$[title=$biblio-title$]$endif$
$endif$

\ifodd\thepage
\blankpage
\fi

%% ========================= Backmatters =============================
\appendix
\backmatter
\cleardoublepage
\pagestyle{appendixpage}
\renewcommand{\chapterlabel}{\appendixname} % 设置附录的页眉

%%------------ 程序代码----------------
\input{./misc/code}

\input{./misc/data}
%% ==================================================================
$for(include-after)$
$include-after$
$endfor$

%%---------------- 致谢 ---------------
\cleardoublepage
\renewcommand{\chapterlabel}{\ackname} % 设置致谢参考文献的页眉
\input{./misc/ack}



\end{document}
